%!TEX root = dissertation-proposal.tex

\section{Data}

\subsection{Study area}

The Kansas City metropolitan area as delineated by the United States Census Bureau is located at 39.0398$^{\circ}$N latitude and 94.5949$^{\circ}$W longitude and spans two states and six counties: Johnson and Wyandotte Counties in Kansas, and Platte, Clay, Cass, and Jackson Counties in Missouri.
The Köppen climate classification is humid subtropical (Cfa), with rainfall year round, averaging 964mm annually. The annual temperature average is 12.8$^{\circ}$C, with a maximum average high of 26.1$^{\circ}$C in July and a minimum average low of -2$^{\circ}$C in January (https://en.climate-data.org/location/715044).
The Kansas City metro area exhibits characteristic patterns of urban sprawl, which is generally defined as “geographic expansion over large areas, low-density land use, low land-use mix, low connectivity, and heavy reliance on automobiles relative to other modes of travel”  \citep{Stone2010} showing a 55 percent increase in built area between 1972 and 2001 \citep{Ji2008}.
The Kansas City metro area  had an estimated population of 2,142,419 in 2018, a 5 percent increase from 2000.
An estimated 24.2\% of the population are under the age of 18.
 73.9\% of the population under the age of 18 are identified as white alone, 7.4\% as black alone, 0.3\% as American Indian alone, 3.2\% as asian alone, 0.4\% as native Hawaiian alone, and 2.6\% as some other race alone.
 6.9\% of the population identify as two or more races and 11.7\% identify as hispanic or latino of any race.
 In 2018, 5.1\% of children under the age of 18 lived in households with income below the poverty level and 7.6\% lived in households receiving some kind of public assistance.
 31.4\% live in single parent households (https://data.census.gov/cedsci).

\subsection{KC Health CORE}

KC Health CORE is a collaborative initiative between Children’s Mercy Hospital and the Center for Economic Information at the University of Missouri, Kansas City created to investigate the geographic disparity of pediatric health outcomes.
This analysis will use pediatric asthma data from 2000-2012 geocoded to street centerlines based on the patients’ home address at the time of admission.
The data come from a retrospective collection of pediatric asthma encounters within the Children’s Mercy Hospital network.
In this instance children ages 2-18 are considered. The original medical records were formatted according to Table \ref{tab:original-asthma-data}.
\begin{table}
	\caption{Structure of the original pediatric asthma data records submitted by CMH to UMKC-CEI. \label{tab:original-asthma-data}}
	\centering
	\begin{tabular}{ll}
		\toprule
		Category & Attributes \\
		\midrule
		Diagnosis & Date of admission \\
		& ICD-9 code \\
		& Event account number \\
		& Patient medical record number (MRN) \\
		& Patient residential address \\
		\midrule
		Demographics & Birthdate \\
		& Sex \\
		& Race \\
		& Ethnicity \\
		\midrule
		Visit characteristics & Payment type \\
		& Patient class \\
		\bottomrule
	\end{tabular}
\end{table}
The data were further classified into three severity levels according to the ICD-9 diagnoses codes (International Classification of Diseases, 9\textsuperscript{th} revision) and the patient class.
The patient class records both the location and the type of treatment received by the patient–e.g. controlled vs. acute care, inpatient vs. outpatient, etc.

\subsection{Pediatric asthma}

Asthma is a collection of symptoms that produce breathing difficulties.
Asthma has been variously shown to be effected by air pressure, temperature, thunderstorms, allergens, and air pollution.
 Asthma occurence has been shown to be higher in individuals living close to highways and railways and other high traffic density areas.
 Asthma occurence also tends to be higher in people of color and among the urban poor.
 Few studies examine the syncronicities between these factors however.

 \subsection{Atmospheric data}

 Atmospheric data were retrieved from the NOAA National Centers for Environmental Information for the Kansas City Downtown Airport, MO, US.
 The station is located at 39.1208$^{\circ}$N, 94.5969$^{\circ}$W. Daily precipitation totals, maximum temperature, and minimum temperature were retrieved for all dates between 1900-01-01 and 2019-10-19.
 Daily average wind speed, direction of fastest 2-minute wind, and Direction of fastest 5-minute wind were retrieved for the years 2000-2012.
 Daily maximum 8-hour ozone concentration and daily mean PM2.5 concentration were retrieved from the EPA for the JFK Community Center in Kansas City, KS, US, located at 39.117219$^{\circ}$N, 94.635605$^{\circ}$W for the years 2000-2012.
 The spatial variation of meteorological data will be assessed using remotely-sensed data including land surface temperature and any other variables available.
 Percentiles within a five-day window across all years were constructed for all variables, as well as the number of days in a row where these variables exceeded extreme percentiles, namely 0.01, 0.05, 0.1, 0.2, 0.8, 0.9, 0.95, and 0.99.
 The diurnal temperature range was also calculated.

 \subsection{Spatial data}

 The Mid-America Regional Council (MARC) created the Natural Resources Inventory (NRI) map of Greater Kansas City with an object-based classification, using SPOT data from May, June, and August of 2012 as well as ancillary data (LiDAR, hydrography, parcels/zoning class, transportation centerlines, streamlines, and floodplains).
 The resulting land cover map has an estimated accuracy of 83 - 91\% for the Level I classifications of impervious, barren, vegetated, and water.
 Impervious comprises buildings and other impervious surfaces, barren comprises land with 0 - 10\% vegetated fraction, vegetated comprises land with 10-100\% vegetated fraction, and water comprises water features.
 The spatial resolution of the NRI landcover map is 2.5 m and the extent is the 4,423 square miles that comprise the 9 county Kansas City metropolitan area \citep{Mid-2013}.

 Additionally, spatial features will be acquired for the study area, including, but not limited to: street and highway network, rail network, and power infrastructure.
 Traffic density data will also be acquired if possible.
 Spatial analysis methods may include overlays to calculate density of infrastructure networks within a set of buffer distances from severe asthma cases, proximity to greenspace, and Bayesian hierarchical spatio-temporal modelling \citep{Khana2018}.
