%!TEX root = dissertation-proposal.tex

\section{Introduction}

\subsection{Overview}
Approximately 55\% of the global population lives in urban areas with this number expected to increase to two-thirds by 2050 \citep{UnitedNations2019}.
Urbanization is accompanied by a suite of surface modifications that effect the surface-energy balance, hydrological flows, and the availability of vegetation, creating distinct and varied microclimates \citep{Pickett2001}.
These modifications vary spatially, creating uneven landscapes of environmental externalities and benefits as well as infrastructure access.
These spatial variations in turn affect and are effected by demographics and are often distinct along income and racial lines \citep{Heynan2006}.
Systematic inequalities arise from zoning practices and disinvestment and have a long historical legacy \citep{Wolch2014}.
These three subsystems–meteorological, physical, and social–are highly interdependent with complex feedbacks, and interact in complex ways characterized by non-linear dynamics and thresholding behaviors, operating across a variety of spatial and temporal scales \citep{Liu2007,Liu2007a,McPhearson2016}.

It is important to understand urban processes and their feedbacks.
Urban systems are complex and incorporate disparate elements usually siloed within separate disciplines \citep{Bai2018}.
It is inefficient to study human and natural systems separately when attempting to undestand their interactions \citep{Liu2007a}.
Likewise, research is usually done in the abstract, without a focused eye to solving real-world problems. Research should provide usable solutions \citep{Liu2007a,McPhearson2016}.
Addressing these problems successfully involves transdisciplinary work that can examine urban systems as intersections of social, physical, environmental, and atmospheric systems \citep{Bai2018,McPhearson2016}.
Understanding urban systems is a critical first step in understanding how they may respond to climate change \citep{Bai2018}.
If risk can be understood quanititatively it can be projected under different climate change scenarios \citep{McMichael2012}.
Measures of public health can be used as an outcome for modelling these coupled human and natural systems.

Public health can be defined as the health of entire populations, from neighborhoods to cities to countries and on \citep{Trochim2006}.
As the world becomes increasingly urban, understanding the dynamics of public health in urban environments and how they interact with climate, the environment, and society becomes increasingly important.
Public health requires transdisciplinary approaches to modelling the complex systems that interact to produce variabilities in outcomes \citep{Trochim2006}.
Understanding the structural variability of public health across these domains can allow for better public policies \citep{McPhearson2016}.

\subsection{Weather and health}

It is intuitive to understand that weather has an effect on human health.
Besides disasters like floods and tornadoes, everyday meteorological conditions effect health as well.
Extremes of heat and cold can exacerbate existing conditions and effect cardiovascular and respiratory function and may lead to increased mortality \citep{ferreirabragaTimeCourseWeatherRelated2001}.
Heat also facilitates the formation of ground-level ozone and precipitation and wind effect the severity of allergen concentrations.
Atmospheric pressure and humidity likewise have effects on human health.
These effects often occur at a temporal lag.

\citet{Ayres-Sampaio2014} found that hospital admissions due to asthma were positively related to high levels of $NO_2$, low NDVI, and high temperatures, however, they examined each of these variables independently in simple linear regressions.
Furthmore, they used seasonal averages of these variables.
Although they attempted a spatial analysis, this consisted of simply examining the differences in relationships between municipalities rather than using a continuously variable measure of space.
\citet{babinPediatricPatientAsthmarelated2007} found a positive relationship between $O_3$ and pediatric asthma but their dataset spanned only three years.

\subsection{Urban infrastructure and environment}

Urban infrastructure and environment vary spatially within the urban context and are likely to produce variabilities in health based on proximity to features.
Highways and railways are major sources of pollution as are power plants and other large electrical facilities.
Greenspace is known to reduce land surface temperature just as impervious surfaces are known to increase it.
\citet{villeneuveCohortStudyRelating2012b} found that there was an inverse relationship between greenspace and mortality.
Urban trees contribute to improved air quality by reducing air pollution concentrations \citep{nowakAirPollutionRemoval2006}.
Urban core areas are effected by the urban heat island where the increased percentage of impervious surfaces elevates temperatures.
Age and material of housing stock are also likely to impact the health of residents.

\subsection{Social determinants of health}

It is well known that large disparities of health outcomes exist across socio-economic spectra, with minorities and the poor having the worst outcomes.
These social determinants of health also vary spatially often along with the physical determinants of health present in the urban system.
It is essential to understand how the social determinants interact with the climatic and physical systems to produce variabilities in public health vulnerability in order to prioritize the distribution of resources.
\citet{babinPediatricPatientAsthmarelated2007} found a logarithmic relationship between pediatric asthma-related emergency departmnet visits and the percentage of children living below the poverty level but this data was aggregated to the zip code ılevel.

\subsection{Current limitations}

While there are many calls for transdisciplinary and systems-based approaches to studying urban areas, few studies have actually attempted to answer the complex questions posed by urban systems.
In particular, public health studies in this vein are even fewer.
While some studies attempt to understand the variability of public health vulnerability in relation to heat and the built environment, the data are too aggregated to get a sense of the actually spatial dependency of these relationships.
Likewise, many studies look at heat-related mortality, however, these counts are not only low enough to be statistically problematic, the mortality coding is problematic as well.
Few, if any, studies incorporate atmospheric, physical, and social systems.

The long-standing paradigm for studying urban systems was to conceptualize them as human systems superimposed upon natural systems.
It has become clear that a more accurate model is that of coupled and natural systems which explicitly characterize the multidirectional and dynamic interactions between these systems.
The modelling of coupled and human natural systems requires statistical techniques that unite data across spatial and temporal scales and can derive meaning over many levels of uncertainty.
Improving these techniques will be key to predicting the effects of climate change on public health.
Understading how human and natural systems are coupled requires modeling the couplings across spatial and temporal scales \cite{Liu2007a}.

\subsection{Research questions}

Here, a set of studies is proposed to examine the variability of public health outcomes:

\begin{enumerate}
	\item What is the relationship between the temporal and spatial variability of public health outcomes and meteorological conditions?
	\item What is the relationship between the spatial variability of public health outcomes and urban infrastructure and environment?
	\item What is the relationship between the temporal and spatial variability of public health outcomes and social determinants of health?
	\item How do social determinants of health interact with climate and urban infrastructure and environment to produce variabilities in public health outcomes?
\end{enumerate}
